	% Copyright 2010 Carlos Contreras Bolton (ccontreras.bolton@gmail.com).
%
% This work may be distributed and/or modified under the
% conditions of the LaTeX Project Public License version 1.3c,
% available at http://www.latex-project.org/lppl/.

\hyphenpenalty=5000
\tolerance=1000
\documentclass[11pt,a4paper]{moderncv}
% moderncv themes
\moderncvtheme{casual}                 % optional argument are 'blue' (default), 'orange', 'red', 'green', 'grey' and 'roman' (for roman fonts, instead of sans serif fonts)
% \moderncvtheme[green]{classic}                % idem
\usepackage{fontawesome}
\usepackage{luatexbase}
\usepackage{microtype}

\usepackage[utf8]{inputenc}                   % replace by the encoding you are using
\usepackage[english, spanish]{babel}
\usepackage{datetime}
\usepackage[numbers]{natbib}
\usepackage[resetlabels]{multibib}

%%%------------------------------------------------------------------
%%% Reverse numbering of bibliography
\usepackage{etaremune}
\usepackage{etoolbox}
\makeatletter
\AtBeginDocument{%%% natbib redefines the environment there
\renewenvironment{thebibliography}[1]
 {\bibsection\parindent\z@\bibpreamble\bibfont
  \settowidth{\dimen0}{#1.}%
  \setlength{\dimen2}{\dimen0}%
  \addtolength{\dimen2}{\labelsep}
  \begin{etaremune}[labelwidth=\dimen0,leftmargin=\dimen2]
  \ifNAT@openbib
    \renewcommand\newblock{\par}%
  \else
    \renewcommand\newblock{\hskip.11em \@plus .33em \@minus .07em}%
  \fi
  \sloppy
  \clubpenalty4000
  \widowpenalty4000
  \sfcode`\.\@m
  \let\NAT@bibitem@first@sw\@firstoftwo
  \let\citeN\cite\let\shortcite\cite\let\citeasnoun\cite}
 {\bibitem@fin\bibpostamble
  \def\@noitemerr{\PackageWarning{natbib}{Empty `thebibliography' environment}}%
  \end{etaremune}
  \bibcleanup}
}%%% end of \AtBeginDocument

%%% patch \@lbibitem to use only \item (for etaremune)
\patchcmd{\@lbibitem}{\item[\hfil\NAT@anchor{#2}{\NAT@num}]}{\item}{}{}
\makeatother

% adjust the page margins
\setlength{\hintscolumnwidth}{3.2cm}                % if you want to change the width of the column with the dates
\usepackage[scale=0.8]{geometry}
\usepackage{gensymb}
\usepackage{footmisc} % enabling footnotes.
\usepackage{academicons}
\usepackage{ifthen}

\recomputelengths                             % required when changes are made to page layout lengths

\definecolor{links}{rgb}{0.22,0.45,0.70}% light blue
%% \definecolor{links}{HTML}{2A1B81}
\renewcommand*{\emailsymbol}         {{\small\faEnvelope}~}
\newcommand{\myurl}[1]{\href{#1}{\color{color1} #1}}
\newcommand{\myurls}[2]{\href{#1}{\color{color1} #2}}
\newcommand{\myurlf}[1]{\href{#1}{\color{color1} \faFilePdf #1}}

\newcommand{\AddressOfficial}{Edmundo Larenas 219, Concepción 4070409, Chile}
\newcommand{\correo}{carlos.contreras.b@udec.cl}
% personal data
\firstname{Carlos}
\familyname{Contreras Bolton}
% \title{Resumé title (optional)}               % optional, remove the line if not wanted
\address{\AddressOfficial}{}    % optional, remove the line if not wanted
% \mobile{mobile (optional)}                    % optional, remove the line if not wanted

% \phone{+39 3387479471}                      % optional, remove the line if not wanted
% \fax{fax (optional)}                          % optional, remove the line if not wanted
\email{\correo}             % optional, remove / comment the line if not wanted
% \extrainfo{additional information (optional)} % optional, remove the line if not wanted
% \photo[64pt]{picture}                         % '64pt' is the height the picture must be resized to and 'picture' is the name of the picture file; optional, remove the line if not wanted
% \quote{"They can kill you, but the legalities of eating you are quite a bit dicier" - David Foster Wallace}                 % optional, remove the line if not wanted

% \nopagenumbers{}                             % uncomment to suppress automatic page numbering for CVs longer than one page
\newcommand\mybitem[1]{%
  \parbox[t]{3mm}{\textbullet}\parbox[t]{14cm}{#1}\\[1.6mm]}

\DeclareOldFontCommand{\rm}{\normalfont\rmfamily}{\mathrm}
\DeclareOldFontCommand{\sf}{\normalfont\sffamily}{\mathsf}
\DeclareOldFontCommand{\tt}{\normalfont\ttfamily}{\mathtt}
\DeclareOldFontCommand{\bf}{\normalfont\bfseries}{\mathbf}
\DeclareOldFontCommand{\it}{\normalfont\itshape}{\mathit}
\DeclareOldFontCommand{\sl}{\normalfont\slshape}{\@nomath\sl}
\DeclareOldFontCommand{\sc}{\normalfont\scshape}{\@nomath\sc}
\DeclareRobustCommand*\cal{\@fontswitch\relax\mathcal}
\DeclareRobustCommand*\mit{\@fontswitch\relax\mathnormal}

\newtoggle{es}
\newcommand{\es}[2]{\iftoggle{es}{#1}{#2}}

\newtoggle{short}
\newcommand{\short}[2]{\iftoggle{short}{#1}{#2}}

\ifthenelse{\equal{\detokenize{cv_es}}{\jobname}}{
  \toggletrue{es}
  \newdateformat{monthyeardate}{%
  \monthname[\THEMONTH] de \THEYEAR}
}{
  \newdateformat{monthyeardate}{%
  \monthname[\THEMONTH] \THEYEAR}
  \togglefalse{es}
}

\ifthenelse{\equal{\detokenize{shortCV_en}}{\jobname}}{
  \toggletrue{short}
}{
  \togglefalse{short}
}

\es{
  \newcites{Sometido,Revista,Libro,Proceeding,Conferencia}{{Artículos sometidos en revistas científicas (bajo revisión)},{Artículos en revistas científicas},{Capítulos de libros},{Conferencias indexadas},{Otras conferencias y charlas internacionales}}
}
{
  \newcites{Sometido,Revista,Libro,Proceeding,Conferencia}{{Journal submitted articles (under revision)},{Journal articles},{Books chapters},{Proceedings of indexed conferences},{Other international conferences and talks}}
}

% ----------------------------------------------------------------------------------
% content
% ----------------------------------------------------------------------------------
\begin{document}
\maketitle
\es {
  \section{Información de contacto}
  {\bf Profesor Asistente}\\
  Departamento de Ingeniería Industrial\\%\footnotemark[1]\\
  Universidad de Concepción\\
  \AddressOfficial
}{
  \section{Contact information}
  {\bf Assistant Professor}\\
  Departamento de Ingeniería Industrial \\%\footnotemark[1]\\
  Universidad de Concepción\\
  \AddressOfficial
}
{
\section{
  \myurls{https://scholar.google.es/citations?user=hIl4ixAAAAAJ\&hl}{\aiGoogleScholar} ~~~~
  \myurls{https://www.webofscience.com/wos/author/record/706245}{\aiPublons} ~~~~
  \myurls{https://www.researchgate.net/profile/Carlos\_Contreras\_Bolton}{\aiResearchGate} ~~~~
  \myurls{https://orcid.org/0000-0001-9549-4143}{\aiOrcid} ~~~~
  \myurls{https://dblp.org/pers/hd/b/Bolton:Carlos\_Contreras}{\aidblp} ~~~~
  \myurls{https://www.linkedin.com/in/carloscontrerasbolton/}{\faLinkedinSquare} ~~~~
  \myurls{https://github.com/contrerasbolton}{\faGithub}
  }
  \faHome~\myurl{https://sites.google.com/view/ccontrerasbolton}\\
  \faEnvelope~\myurl{\correo}\\
  \faPhone~+56 41 2203616
  }

  \es {
    \selectlanguage{spanish}
    \footnotetext[1]{Última actualización en \monthyeardate\today}
  }{
    \selectlanguage{english}
    \footnotetext[1]{Last updated on \monthyeardate\today}
  }
  \es{
    \section{Posición actual}
    \cventry{Desde enero 2020}{Profesor Asistente}{Departamento de Ingeniería Industrial}{Universidad de Concepción.}{}{}
  }{
    \section{Current position}
    \cventry{From January 2020}{Assistant Professor}{Departamento de Ingeniería Industrial}{Universidad de Concepción.}{}{}
  }

\es {
  \section{Formación académica}
  \cventry{2016 -- 2018}{Dottore di Ricerca in Ingegneria Biomedica, Elettrica e dei Sistemi}{Doctor en Ingeniería Biomédica, Eléctrica y de Sistemas}{Università di Bologna}{Italia. Especialización en Investigación de Operaciones.}{
    Tesis: ``Algorithms for Variants of Routing Problems''. \newline
    Supervisores: Paolo Toth - Valentina Cacchiani - Daniele Vigo %La defensa tendrá lugar en marzo 2019
}
  \cventry{2011 -- 2013}{Magíster en Ingeniería Informática}{Universidad de Santiago de
    Chile.}{}{}{Tesis: ``Combinación Automática de Operadores de un Algoritmo Genético
    para resolver el Problema del Vendedor Viajero Simétrico''.\newline
    Supervisor: Víctor Parada}
  \cventry{2005 -- 2010}{Licenciado en Ciencia de la Computación}{Analista en Computación Científica}{Universidad de Santiago de Chile.}{}{Tesis: ``Generación Automática de Algoritmos para el problema de Coloración de Vértices''.\newline
    Supervisores: Rúben Carvajal Schiaffino - Víctor Parada}
  %% \cventry{2001 -- 2004}{Enseñanza Media}{Liceo Manuel Barros Borgoño}{Santiago, Chile.}{}{}
}
{
  \section{Education}
  \cventry{2016 -- 2018}{Ph.D. in Biomedical, Electrical and System Engineering}{Dottore di Ricerca in Ingegneria Biomedica, Elettrica e dei Sistemi}
  {Università di Bologna}
  {Italy. Majoring in Operations Research.}{
    Thesis: ``Algorithms for Variants of Routing Problems''.  \newline
    Advisors: Paolo Toth - Valentina Cacchiani - Daniele Vigo
  }
  \cventry{2011 -- 2013}{M.Sc. in Computer Science Engineering}{Universidad
    de Santiago de Chile.}{}{}{Thesis: ``Automatic Combination of Operators in a
    Genetic Algorithm to solve the Symmetric Traveling Salesman Problem''.\newline
    Advisor: Víctor Parada}
  \cventry{2005 -- 2010}{B.Sc. in Computer Science}{Professional Degree in
    Analyst in Scientific Computing}{Universidad de Santiago de Chile.}{}{Thesis: ``Automatic Generation of Algorithms for the Vertex Coloring Problem''.\newline
    Advisors: Rúben Carvajal Schiaffino - Víctor Parada}
}
\es{
  \section{Proyectos de investigación}
    \cventry{2024 -- 2026}{FONDECYT Iniciación (ANID)}{Automatic generation of metaheuristic algorithms for variants of the traveling salesman problem}{Código 11241132}{{\bf Investigador Principal}.}{}
    \cventry{2023 -- 2026}{FONDECYT Regular (ANID)}{Trajectory-based metaheuristic algorithms for new variants of the VRP}{Código 1230125}{{\bf Co-investigador}.}{}
    \cventry{2020 -- 2022}{VRID Iniciación (UDEC)}{Generación automática de algoritmos metaheurísticos para variantes del problema del vendedor viajero}{Código 220.097.016-INI}{{\bf Investigador Principal}.}{}
}{
  \section{Research grants} 
      \cventry{2024 -- 2026}{FONDECYT Iniciación (ANID)}{Automatic generation of metaheuristic algorithms for variants of the traveling salesman problem}{Project number 11241132}{{\bf Principal investigator}.}{}
      \cventry{2023 -- 2026}{FONDECYT Regular (ANID)}{Trajectory-based metaheuristic algorithms for new variants of the VRP}{Project number 1230125}{{\bf Co-investigator}.}{}
      \cventry{2020 -- 2022}{VRID Iniciación (UDEC)}{Automatic generation of metaheuristic algorithms for variants of the traveling salesman problem}{Project number 220.097.016-INI}{{\bf Principal investigator}.}{}
}
\newpage
\es
{
  \section{Posiciones pasadas}
  \cventry{marzo 2019 -- enero 2020}{Investigador Postdoctoral y Profesor part-time}{Departamento de Ingeniería Informática}{Universidad de Santiago de Chile.}{}{}
  \cventry{noviembre 2015 -- octubre 2018}{Estudiante de Doctorado}{Dottorato di Ricerca in Ingegneria Biomedica, Elettrica, dei Sistemi}{Dipartimento
  di Ingegneria dell’Energia Elettrica e dell’Informazione ``Guglielmo Marconi''}{Università di Bologna.}{}
  \cventry{julio 2015 -- octubre 2015}{Desarrollador de software y Asistente de Investigación}{Instituto Sistemas Complejos de Ingeniería.}{}{}{}
  \cventry{agosto 2012 -- septiembre 2015}{Profesor part-time}{Departamento de Ingeniería Informática}{Universidad de Santiago de Chile.}{}{}
  \cventry{marzo 2012 -- julio 2015}{Profesor part-time}{Facultad de Ingeniería}{Universidad Andrés Bello.}{}{}
  \cventry{marzo 2012 -- diciembre 2012}{Ayudante}{Ingeniería Civil en Informática}{Departamento de Ingeniería Informática}{Universidad de Santiago de Chile.}{}
  \cventry{agosto 2011 -- agosto 2012}{Ayudante}{Facultad de Ingeniería}{Universidad Andrés Bello.}{}{}
  \cventry{mayo 2010 -- febrero 2011}{Desarrollador de sistema}{SubDirección de Operación y Plataforma}{Dirección de Informática}{Pontificia Universidad Católica de Chile.}{}{}{}
  \cventry{marzo 2010 -- diciembre 2010}{Ayudante}{Departamento de Ingeniería Metalúrgica}{Universidad de Santiago de Chile.}{}{}
  \cventry{julio 2009 -- diciembre 2013}{Ayudante de Investigación}{Grupo de Optimización}{Departamento de Informática}{Universidad de Santiago de Chile.}{}
}
{
  \section{Past positions}
  \cventry{March 2019 -- January 2020}{Postdoctoral Researcher and Lecturer}{Departamento de Ingeniería Informática}{Universidad de Santiago de Chile.}{}{}
  \cventry{November 2015 -- October 2018}{Ph.D. Student}{Ph.D. Programme Biomedical, Electrical and System Engineering}{Dipartimento
  di Ingegneria dell’Energia Elettrica e dell’Informazione ``Guglielmo Marconi''}{Università di Bologna.}{}
  \cventry{July 2015 -- October 2015}{Software Developer and Research Assistant}{Instituto Sistemas Complejos de Ingeniería.}{}{}{}
  \cventry{August 2012 -- September 2015}{Part-time Lecturer}{Departamento de Ingeniería Informática}{Universidad de Santiago de Chile.}{}{}
  \cventry{March 2012 -- July 2015}{Part-time Lecturer}{Facultad de Ingeniería}{Universidad Andrés Bello.}{}{}
  \cventry{March 2012 -- December 2012}{Teaching Assistant}{Departamento de Ingeniería Informática}{Universidad de Santiago de Chile.}{}{}
  \cventry{August 2011 -- August 2012}{Teaching Assistant}{Facultad de Ingeniería}{Universidad Andrés Bello.}{}{}{}
  \cventry{May 2010 -- February 2011}{System Developer}{Section of Operation and Platform}{Dirección de Informática}{Pontificia Universidad Católica de Chile.}{}
  \cventry{March 2010 -- December 2010}{Teaching Assistant}{Departamento de Ingeniería Metalúrgica}{Universidad de Santiago de Chile.}{}{}
  \cventry{July 2009 -- December 2013}{Research Assistant in Optimization Group}{Departamento de Ingeniería Informática}{Universidad de Santiago de Chile.}{}{}
}
\short{
}
{
\es{
\section{Nombramientos docentes y académicos}
%\cventry{Desde 2023}{Profesor Claustro}{Magíster en Ciencias de la Computación}{Departamento de Ingeniería Informática y Ciencias de la Computación}{Universidad de Concepción.}{} % 0 años, desde XX/09/2023.
\cventry{Desde 2022}{Profesor Colaborador}{Doctorado en Ingeniería Industrial}{Departamento de Ingeniería Industrial}{Universidad de Concepción.}{} % 2 años, desde 07/03/2022.
\cventry{Desde 2021}{Profesor Claustro}{Magíster en Ingeniería Industrial}{Departamento de Ingeniería Industrial}{Universidad de Concepción.}{} % 3 años, desde 02/03/2021.
%\cventry{Desde 2021}{Profesor Claustro}{Magíster en Gestión Industrial}{Departamento de Ingeniería Industrial}{Universidad de Concepción.}{} % 2 años, desde 21/09/2021.
\cventry{Desde 2020}{Profesor Asistente}{Departamento de Ingeniería Industrial}{Universidad de Concepción.}{}{}

\section{Servicios docentes}
\cventry{Desde 2020}{Miembro del comité de Magíster en Ingeniería Industrial}{Departamento de Ingeniería Industrial}{Universidad de Concepción.}{}{}
\cventry{Desde 2022}{Miembro del comité de carrera de Ingeniería Industrial}{Departamento de Ingeniería Industrial}{Universidad de Concepción.}{}{}

\section{Preparación en docencia}
    \cventry{2021}{Diplomado en formación, aprendizaje y evaluación en educación superior}{Universidad de Concepción.}{}{}{}
\newpage
\section{Experiencia en docencia}
  \subsection{Universidad de Concepción}
    \cventry{2022}{Introducción a la Innovación en Ingeniería}{a varias Ingenierías Civiles}{Facultad de Ingeniería}{impartida como Profesor en 2022-2.}{}
    \cventry{2020}{Algoritmos Modernos para Producción}{Doctorado y Magíster en Ingeniería Industrial}{Departamento de Ingeniería Industrial}{impartida conjuntamente con la Prof. Rosa Medina y Prof. Lorena Pradenas en 2020-2, 2021-2, 2022-2 y 2023-2.}{}
    \cventry{2020}{Modelos Determinísticos en Investigación de Operaciones}{Doctorado y Magíster en Ingeniería Industrial}{Departamento de Ingeniería Industrial}{impartida conjuntamente con la Prof. Rosa Medina y la Prof. Lorena Pradenas en 2020-1, 2021-1, 2022-1 y 2023-1.}{}
    \cventry{2020}{Programación Aplicada a la Ingeniería Industrial}{Ingeniería Civil Industrial}{Departamento de Ingeniería Industrial}{impartida como Profesor en 2020-2, 2021-1, 2021-2, 2022-1, 2022-2, 2023-1 y 2023-2.}{}
    \cventry{2020}{Optimización I}{Ingeniería Civil Industrial}{Departamento de Ingeniería Industrial}{impartida conjuntamente con la Prof. Rosa Medina en 2020-1, 2021-1, 2022-1 y 2023-1.}{}
    \cventry{2020}{Optimización II}{Ingeniería Civil Industrial}{Departamento de Ingeniería Industrial}{impartida conjuntamente con la Prof. Rosa Medina en en 2020-2, 2021-2, 2022-2 y 2023-1.}{}

  \subsection{Universidad de Santiago de Chile}
    \cventry{2019}{Modelos y Algoritmos de Optimización}{Licenciatura en Ciencia de la Computación}{Departamento de Matemática y Ciencia de la Computación}{impartida como Profesor en 2019-1 y 2019-2.}{}
    \cventry{2015}{Métodos de Programación}{Ingeniería Civil en Informática}{Departamento de Ingeniería en Informática}{impartida como Profesor en 2015-1.}{}
    \cventry{2012}{Fundamentos de Computación y Programación}{A varias Ingenierías de Ejecución y Civil}{Facultad de Ingeniería}{impartida como Profesor en 2012-2, 2013-2, 2014-1, 2014-2, 2015-1 y 2019-1.}{}
    \cventry{2012}{Computación Evolutiva}{Ingeniería Civil en Informática}{Departamento de Ingeniería Informática}{impartida como Ayudante en 2012-1 y 2012-2.}{}
    \cventry{2010}{Sistemas de Información y Comunicación}{Ingeniería de Ejecución en Metalurgia}{Departamento de Ingeniería Metalúrgica}{impartida como Ayudante en 2010-1 y 2010-2.}{}
  \subsection{Universidad Andrés Bello}
    \cventry{2014}{Ingeniería de Software II}{Ingeniería Civil en Informática}{Facultad de Ingeniería}{impartida como Profesor en 2014-2.}{}
    \cventry{2014}{Programación I}{Ingeniería en Computación e Informática}{Facultad de Ingeniería}{impartida como Profesor en 2014-1 y 2015-1.}{}
    \cventry{2014}{Programación II}{Ingeniería en Computación e Informática}{Facultad de Ingeniería}{impartida como Profesor en 2014-2.}{}
    \cventry{2013}{Análisis y Diseño de Algoritmos}{Ingeniería Civil en Informática}{Facultad de Ingeniería}{impartida como Profesor en 2013-2.}{}
    \cventry{2013}{Fundamentos de Programación}{Ingeniería Civil Informática}{Facultad de Ingeniería}{impartida como Profesor en 2013-2 y 2014-2.}{}
    \cventry{2013}{Algoritmos y Estructura de Datos}{Ingeniería Civil Informática}{Facultad de Ingeniería}{impartida como Profesor en 2013-1 y 2014-1.}{}
    \cventry{2013}{Proyecto de Título I}{Ingeniería en Computación e Informática}{Facultad de Ingeniería}{impartida como Profesor en 2013-1, 2013-2, 2014-2 y 2015-1.}{}
    \cventry{2013}{Proyecto de Título II}{Ingeniería en Computación e Informática}{Facultad de Ingeniería}{impartida como Profesor en 2013-1, 2013-2, 2014-1 y 2014-2.}{}
    \cventry{2013}{Estructura de Datos}{Ingeniería en Computación e Informática}{Facultad de Ingeniería}{impartida como Profesor en 2013-1, 2014-1, 2014-2 y 2015-1.}{}
    \cventry{2013}{Diseño de Algoritmos}{Ingeniería en Computación e Informática}{Facultad de Ingeniería}{impartida como Profesor en 2013-1, 2013-2, 2014-1 y 2015-1.}{}
    \cventry{2013}{Metodología de Desarrollo de Software}{Ingeniería en Computación e Informática}{Facultad de Ingeniería}{impartida como Profesor en 2013-1.}{}
    \cventry{2012}{Investigación de Operaciones}{Ingeniería en Computación e Informática}{Facultad de Ingeniería}{impartida como Profesor en 2012-2.}{}
    \cventry{2012}{Metaheurísticas}{Ingeniería Civil en Informática}{Facultad de Ingeniería}{impartida como Profesor en 2012-2.}{}
    \cventry{2012}{Tecnologías de la Información}{Ingeniería Civil Industrial}{Facultad de Ingeniería}{impartida como Profesor en 2012-2.}{}
    \cventry{2011}{Investigación de Operaciones}{Ingeniería en Computación e Informática}{Facultad de Ingeniería}{impartida como Ayudante en 2011-2 y 2012-1.}{}

}{

\section{Teaching and academic appointments}
%\cventry{From 2023}{Claustro Professor}{M.Sc. in Computer Science}{Departamento de Ingeniería Informática y Ciencias de la Computación}{Universidad de Concepción.}{} % 0 años, desde XX/09/2023.
\cventry{From 2022}{Collaborator Professor}{Ph.D. in Industrial Engineering}{Departamento de Ingeniería Industrial}{Universidad de Concepción.}{}
\cventry{From 2021}{Claustro Professor}{M.Sc. in Industrial Engineering}{Departamento de Ingeniería Industrial}{Universidad de Concepción.}{}
%\cventry{From 2021}{Claustro Professor}{M.Sc. in Industrial Management}{Departamento de Ingeniería Industrial}{Universidad de Concepción.}{}
\cventry{From 2020}{Assistant Professor}{Departamento de Ingeniería Industrial}{Universidad de Concepción.}{}{}



\section{Teaching service}
\cventry{From 2020}{Member of the Committee of M.Sc. in Industrial Engineering}{Departamento de Ingeniería Industrial}{Universidad de Concepción.}{}{}
\cventry{From 2022}{Member of the Committee of Industrial Engineering}{Departamento de Ingeniería Industrial}{Universidad de Concepción.}{}{}

\section{Teaching preparation}
    \cventry{2021}{Diploma in training, learning and evaluation in higher education}{Universidad de Concepción.}{}{}{}
\newpage
\section{Teaching experience}
\subsection{Universidad de Concepción}
    \cventry{2022}{Introduction to Engineering Innovation}{to several Engineering}{Facultad de Ingeniería}{given as Professor in 2022-2.}{}
    \cventry{2020}{Modern Algorithms for Production}{Ph.D. and M.Sc. in Industrial Engineering}{Departamento de Ingeniería Industrial}{given jointly with Prof. Rosa Medina and Prof. Lorena Pradenas in 2020-2, 2021-2, 2022-2 and 2023-2.}{}
    \cventry{2020}{Deterministic Models in Operations Research}{Ph.D. and M.Sc. in Industrial Engineering}{Departamento de Ingeniería Industrial}{given jointly with Prof. Rosa Medina and Prof. Lorena Pradenas in 2020-1, 2021-1, 2022-1 and 2023-1.}{}
    \cventry{2020}{Programming Applied to Industrial Engineering}{Industrial Engineering}{Departamento de Ingeniería Industrial}{given as Professor in 2020-2, 2021-1, 2021-2, 2022-1, 2022-2, 2023-1 and 2023-2.}{}
    \cventry{2020}{Optimization I}{Industrial Engineering}{Departamento de Ingeniería Industrial}{given jointly with Prof. Rosa Medina in 2020-1, 2021-1, 2022-1 and 2023-1.}{}
    \cventry{2020}{Optimization II}{Industrial Engineering}{Departamento de Ingeniería Industrial}{given jointly with Prof. Rosa Medina in 2020-2, 2021-2, 2022-2 and 2023-2.}{}
    \subsection{Universidad de Santiago de Chile}
    \cventry{2019}{Models and Algorithms for Optimization}{B.Sc. in Computer Science}{Departamento de Matemática y Ciencia de la Computación}{given as Professor in 2019-1 and 2019-2.}{}
    \cventry{2015}{Programming Methods}{Computer Science Engineering}{Departamento de Ingeniería en Informática}{given as Professor in 2015-1.}{}
    \cventry{2012}{Computer Fundamentals and Programming}{to several Engineering}{Facultad de Ingeniería}{given as Professor in 2012-2, 2013-2, 2014-1, 2014-2, 2015-1 and 2019-1.}{}
    \cventry{2012}{Evolutionary Computation}{Computer Science Engineering}{Departamento de Ingeniería Informática}{given as Teaching Assistant in 2012-1 and 2012-2.}{}
    \cventry{2010}{Information and Communication Systems}{Metallurgical Engineering}{Departamento de Ingeniería Metalúrgica}{given as Teaching Assistant in 2010-1 and 2010-2.}{}
    \subsection{Universidad Andrés Bello}
    \cventry{2014}{Software Engineering II}{Computer Science Engineering}{Facultad de Ingeniería}{given as Professor in 2014-2.}{}
    \cventry{2014}{Programming I}{Computer Science and Informatics Engineering}{Facultad de Ingeniería}{given as Professor in 2014-1 and 2015-1.}{}
    \cventry{2014}{Programming II}{Computer Science and Informatics Engineering}{Facultad de Ingeniería}{given as Professor in 2014-2.}{}
    \cventry{2013}{Analysis and Design of Algorithms}{Computer Science Engineering}{Facultad de Ingeniería}{given as Professor in 2013-2.}{}
    \cventry{2013}{Programming Fundamentals}{Computer Science Engineering}{Facultad de Ingeniería}{given as Professor in 2013-2 and 2014-2.}{}
    \cventry{2013}{Algorithms and Data Structures}{Computer Science Engineering}{Facultad de Ingeniería}{given as Professor in 2013-1 and 2014-1.}{}
    \cventry{2013}{Degree Project I}{Computer Science and Informatics Engineering}{Facultad de Ingeniería}{given as Professor in 2013-1, 2013-2, 2014-2 and 2015-1.}{}
    \cventry{2013}{Degree Project II}{Computer Science and Informatics Engineering}{Facultad de Ingeniería}{given as Professor in 2013-1, 2013-2, 2014-1 and 2014-2.}{}
    \cventry{2013}{Data Structures}{Computer Science and Informatics Engineering}{Facultad de Ingeniería}{given as Professor in 2013-1, 2014-1, 2014-2 and 2015-1.}{}
    \cventry{2013}{Algorithms Design}{Computer Science and Informatics Engineering}{Facultad de Ingeniería}{given as Professor in 2013-1, 2013-2, 2014-1 and 2015-1.}{}
    \cventry{2013}{Software Development Methodologies}{Computer Science and Informatics Engineering}{Facultad de Ingeniería}{given as Professor in 2013-1.}{}
    \cventry{2012}{Operations Research}{Computer Science and Informatics Engineering}{Facultad de Ingeniería}{given as Professor in 2012-2.}{}
    \cventry{2012}{Metaheuristics}{Computer Science Engineering}{Facultad de Ingeniería}{given as Professor in 2012-2.}{}
    \cventry{2012}{Information Technology}{Industrial Engineering}{Facultad de Ingeniería}{given as Professor in 2012-2.}{}
    \cventry{2011}{Operations Research}{Computer Science and Informatics Engineering}{Facultad de Ingeniería}{given as Teaching Assistant in 2011-2 and 2012-1}{}
}
} %% end short

%% Load supervision file
\es{
%\section{Memorias de título y tesis supervisadas}
{
\def\thing{0}

  \newcommand{\PTp}{2}  %% Memorias de título en progreso
  \newcommand{\PTf}{9}  %% Memorias de título finalizadas
  \newcommand{\MTp}{0}  %% Tesis de magíster en progreso
  \newcommand{\MTf}{2}  %% Tesis de magíster finalizadas
  \newcommand{\cMTf}{1} %% Tesis de co-guía magíster finalizadas
  \newcommand{\DTp}{0}  %% Tesis de doctorado en progreso
  \newcommand{\DTf}{0}  %% Tesis de doctorado finalizadas
  \newcommand{\cDTp}{1}  %% Tesis de co-guía doctorado en progreso


\newcommand{\printStudents}[2]{\ifthenelse{\equal{#1}{1}}{#1 #2 student}{#1 #2 students}}
\newcommand{\printEstudiantes}[2]{\ifthenelse{\equal{#1}{1}}{#1 estudiante de #2}{#1 estudiantes de #2}}

%\printEstudiantes{\PTp}
\es{
    \cvitem{En progreso}{\printEstudiantes{\PTp{}}{pregrado}\ifthenelse{\equal{\thing}{\cDTp}}{}{ y \printEstudiantes{\cMTf}{doctorado co-guíado}}\ifthenelse{\equal{\thing}{\MTp}}{}{y \printEstudiantes{\MTp}{magíster}}.}
    \cvitem{Finalizadas}{\printEstudiantes{\PTf{}}{pregrado}, \printEstudiantes{\MTf}{magíster} y \printEstudiantes{\cMTf}{magíster co-guíado}.}
  }{
    \cvitem{In progress}{\printStudents{\PTp{}}{undergraduate}\ifthenelse{\equal{\thing}{\cDTp}}{}{ and \printStudents{\cMTf}{co-supervisor Ph.D.}}\ifthenelse{\equal{\thing}{\MTp}}{}{and \printStudents{\MTp}{M.Sc.}}.}
    \cvitem{Finished}{\printStudents{\PTf{}}{undergraduate}, \printStudents{\MTf}{M.Sc.} and \printStudents{\cMTf}{co-supervisor M.Sc.}.}
  }
} %% no quitar, las referencias se comienzan a juntar palabras

%\section{Comisión evaluadora de memorias de título y tesis}
%  \cventry{}{}{29 Memorias de título y 16 tesis de magíster.}{}{}{}
}{
%\section{Supervised final undergraduate projects and theses}
{
\def\thing{0}

  \newcommand{\PTp}{2}  %% Memorias de título en progreso
  \newcommand{\PTf}{9}  %% Memorias de título finalizadas
  \newcommand{\MTp}{0}  %% Tesis de magíster en progreso
  \newcommand{\MTf}{2}  %% Tesis de magíster finalizadas
  \newcommand{\cMTf}{1} %% Tesis de co-guía magíster finalizadas
  \newcommand{\DTp}{0}  %% Tesis de doctorado en progreso
  \newcommand{\DTf}{0}  %% Tesis de doctorado finalizadas
  \newcommand{\cDTp}{1}  %% Tesis de co-guía doctorado en progreso


\newcommand{\printStudents}[2]{\ifthenelse{\equal{#1}{1}}{#1 #2 student}{#1 #2 students}}
\newcommand{\printEstudiantes}[2]{\ifthenelse{\equal{#1}{1}}{#1 estudiante de #2}{#1 estudiantes de #2}}

%\printEstudiantes{\PTp}
\es{
    \cvitem{En progreso}{\printEstudiantes{\PTp{}}{pregrado}\ifthenelse{\equal{\thing}{\cDTp}}{}{ y \printEstudiantes{\cMTf}{doctorado co-guíado}}\ifthenelse{\equal{\thing}{\MTp}}{}{y \printEstudiantes{\MTp}{magíster}}.}
    \cvitem{Finalizadas}{\printEstudiantes{\PTf{}}{pregrado}, \printEstudiantes{\MTf}{magíster} y \printEstudiantes{\cMTf}{magíster co-guíado}.}
  }{
    \cvitem{In progress}{\printStudents{\PTp{}}{undergraduate}\ifthenelse{\equal{\thing}{\cDTp}}{}{ and \printStudents{\cMTf}{co-supervisor Ph.D.}}\ifthenelse{\equal{\thing}{\MTp}}{}{and \printStudents{\MTp}{M.Sc.}}.}
    \cvitem{Finished}{\printStudents{\PTf{}}{undergraduate}, \printStudents{\MTf}{M.Sc.} and \printStudents{\cMTf}{co-supervisor M.Sc.}.}
  }
} %% no quitar, las referencias se comienzan a juntar palabras

%\section{Final undergraduate projects and theses assessment committee}
%  \cventry{}{}{29 final undergraduate projects and 16 M.Sc. theses.}{}{}{}
}

\es{
%% \newpage
  \section{Premios y reconocimientos}
  \cventry{noviembre 2015}{Beca Doctorado en el extranjero - Beca Chile
    2015}{Comisión Nacional de Investigación Científica y Tecnológica
    (CONICYT).}{}{}{Beca para realizar estudios conducentes a la obtención del
    grado académico de Doctor en instituciones de excelencia en el exterior.}
  \cventry{julio 2013}{Google Scholars' Retreat}{Google.}{}{}{Como becario de
    Google Latam me invitaron a la oficina de Google en Nueva York con otros
    becarios de América Latina y EE.UU. para celebrar los logros, interactuar
    con otros estudiosos, y escuchar charlass de los ingenieros de Google.}
  \cventry{marzo 2013}{Google USACH Scholarship}{Google.}{}{}{Premiado con la
    beca Google en el marco del convenio Google University Programs que busca
    recompensar a estudiantes de diversos orígenes para incentivar la
    excelencia en los estudios y promover que se conviertan en modelos de rol y
    líderes activos en el campo de la ingeniería informática.}
}{
%%\newpage
  \section{Honors and awards}
  \cventry{November 2015}{Ph.D. scholarship abroad - Beca Chile 2015}{Comisión
    Nacional de Investigación Científica y Tecnológica
    (CONICYT).}{}{}{Scholarship for studying the academic degree of Doctor of institutions of excellence abroad.}
  \cventry{July 2013}{Google Scholars' Retreat}{Google.}{}{}{As a Google Latam
    scholarship recipient I was invited to a Google office in NYC with other
    recipients from Latin America and USA to celebrate accomplishments,
    interact with fellow scholars, and hear talks from Google engineers.}
  \cventry{March 2013}{Google Latin America Scholarship}{Google.}{}{}{Scholarships awarded
    based on the strength of academic background and demonstrated passion for
    Computer Science.}
}
\short
{
}{
\es{
  \section{Asistencias a escuelas}
  \cventry{enero 2017}{Winter School on Network Optimization $6^{th}$ edition}{Centro de Matemática,
    Aplicações Fundamentais e Investigação Operacional}{Faculdade de Ciências, Universidade de
    Lisboa}{Estoril, Portugal.}{Conferencias de Ángel Corberán, Arie Koster, Gerhard Reinelt, Giovanni
    Rinaldi y José Valério de Carvalho.}

  \cventry{agosto 2016}{Prague Summer Schools on Discrete Mathematics}{Mathematical Institute of the
    Czech Academy of Sciences and Computer Science Institute of Charles University}{}{Praga, República
    Checa}{Conferencias de Ronald de Wolf y Samuel Fiorini.}

  \cventry{junio 2016}{COST/MINO Ph.D. School on Advanced Optimization Methods 2016}{Istituto di Analisi
    dei Sistemi ed Informatica ``A. Ruberti''}{Consiglio Nazionale delle Ricerche}{Roma,
    Italia}{Conferencias de Santanu S. Dey, Jordi Castro, Antonio Frangioni y Veronica Piccialli.}
}
{
  \section{Schools attended}
  \cventry{January 2017}{Winter School on Network Optimization $6^{th}$ edition}{Centro de Matemática,
    Aplicações Fundamentais e Investigação Operacional}{Faculdade de Ciências, Universidade de
    Lisboa}{Estoril, Portugal.}{Lectures by Ángel Corberán, Arie Koster, Gerhard Reinelt, Giovanni
    Rinaldi and José Valério de Carvalho.}

  \cventry{August 2016}{Prague Summer Schools on Discrete Mathematics}{Mathematical Institute of the
    Czech Academy of Sciences and Computer Science Institute of Charles University}{}{Prague, Czech
    Republic}{Lectures by Ronald de Wolf and Samuel Fiorini.}

  \cventry{June 2016}{COST/MINO Ph.D. School on Advanced Optimization Methods 2016}{Istituto di Analisi
    dei Sistemi ed Informatica ``A. Ruberti''}{Consiglio Nazionale delle Ricerche}{Roma,
    Italy}{Lectures by Santanu S. Dey, Jordi Castro, Antonio Frangioni and Veronica Piccialli.}
}

\es
{
  \section{Posiciones de visitante}
  \cventry{diciembre 2018}{Universidad del Bío-Bío.}{}{}{}{Investigador Visitante en el Departmento de
    Industrial Engineering Ingeniería Industrial. Trabajando con Rodrigo Linfati.}
}
{
  \section{Visiting positions}
  \cventry{December 2018}{Universidad del Bío-Bío.}{}{}{}{Visiting Researcher at the Department of
    Industrial Engineering. Working with Rodrigo Linfati.}
}

\es
{
  \section{Charlas invitadas}
  \cventry{enero 2022}{``¿Cómo las matemáticas mejoran nuestra vida?: Optimizando la realidad''.}{}{}{}{Explora RM Sur Oriente.}
  \cventry{diciembre 2018}{``Algorithms for Variants of Routing Problems''.}{}{}{}{Universidad de
    Santiago de Chile, Departamento de Matemática y Ciencia de la Computación.}
}
{
  \section{Invited talks}
  \cventry{January 2022}{``How does mathematics improve our lives?: Optimizing reality''.}{}{}{}{Explora RM Sur Oriente.}
  \cventry{December 2018}{``Algorithms for Variants of Routing Problems''.}{}{}{}{Universidad de
    Santiago de Chile, Department of Mathematics and Computer Science.}
}

\es
{
  \section{Miembro de comités}
  \cventry{2020}{Miembro del comité Científico.}{}{}{}{International Conference of Production Research, ICPR - Americas 2020, Bahía Blanca, Argentina.}
  \cventry{2018}{Miembro del comité de Organización.}{}{}{}{EURO/ALIO International Conference 2018 on Applied Combinatorial Optimization, Bologna,
  Italia.}
}
{
  \section{Academic duties}
  \cventry{2020}{Member of the Scientific Committee.}{}{}{}{International Conference of Production Research, ICPR - Americas 2020, Bahía Blanca, Argentina.}
  \cventry{2018}{Member of the Organizing Committee.}{}{}{}{EURO/ALIO International Conference 2018 on Applied Combinatorial Optimization, Bologna,
  Italy.}
}

\es{
%\newpage
  \section{Afiliaciones}
  \cventry{July 2023 -- Present}{Association for Computing Machinery (ACM)}{Sociedad de computación más grande del mundo}{Miembro.}{}{}
  \cventry{marzo 2012 -- a la fecha}{Instituto Chileno de Investigación de Operaciones (ICHIO)}{Sociedad Chilena de Investigación Operativa}{Miembro.}{}{}
  \cventry{septiembre 2016 -- diciembre 2018}{Associazione Italiana di Ricerca Operativa (AIRO)}{Sociedad Italiana de Investigación Operativa}{Miembro.}{}{}
  \cventry{agosto 2008 -- marzo 2015}{POS RIZOMA COMERCIO}{Proyecto de Software Libre}{Software de Gestión de Pymes}{Desarrollador.}{}
}
{
%%\newpage
  \section{Affiliations}
  \cventry{July 2023 -- Present}{Association for Computing Machinery (ACM)}{The world's largest computing society}{Member.}{}{}
  \cventry{March 2012 -- Present}{Instituto Chileno de Investigación de Operaciones (ICHIO)}{Chilean Operations Research Society}{Member.}{}{}
  \cventry{September 2016 -- December 2018}{Associazione Italiana di Ricerca Operativa (AIRO)}{Italian Operations Research Society}{Member.}{}{}
  \cventry{August 2008 -- March 2015}{POS RIZOMA COMERCIO}{Project of Free Software}{SMEs Management Software}{Developer.}{}
}
}

\newcommand{\programming}[1]{C/C++, Java, Python, Julia, HTML, PHP, JSP, Bash, Perl, Ruby {#1} Javascript.}
\newcommand{\software}[1]{LaTeX, Matlab, Octave, Maple, R, Gurobi, Cplex {#1} AMPL.}
\newcommand{\library}[1]{GTK, MPI, OpenMP, Django {#1} Pthreads.}
\es
{
  %% \newpage
  \section{Conocimientos computacionales}
  \cvitem{Programación}{\programming{y}}
  \cvitem{Bibliotecas}{\library{y}}
  \cvitem{Software Científico}{\software{y}}
  %% \cvitem{Ofimática}{OpenOffice/LibreOffice/Microsoft Office.}
  %% \cvitem{Bases de Datos}{PostgresSQL, Mysql, Sqlite y Oracle.}
}
{
 %% \newpage
  \section{Computer skills}
  \cvitem{Programming}{\programming{and}}
  \cvitem{Libraries}{\library{and}}
  \cvitem{Scientific Software}{\software{and}}
  %% \cvitem{Office suite}{OpenOffice/LibreOffice/Microsoft Office.}
  %% \cvitem{Database}{PostgresSQL, Mysql, Sqlite and Oracle.}
}
\es{
  \section{Idiomas}
  \cventry{Español}{Lengua materna}{}{}{}{}
  \cventry{Inglés}{Bueno}{}{}{}{}
  \cventry{Italiano}{Bueno}{}{}{}{}
}
{
  \section{Languages}
  \cventry{Spanish}{Mother tongue}{}{}{}{}
  \cventry{English}{Good}{}{}{}{}
  \cventry{Italian}{Good}{}{}{}{}
}


%% Load reviews file
%% File with the reviews
%% Journals
\newcommand{\EJOR}{\myurls{https://www.journals.elsevier.com/european-journal-of-operational-research}{European Journal of Operational Research}}   		%% 1
\newcommand{\INS}{\myurls{https://www.journals.elsevier.com/information-sciences}{Information Sciences}}                                            		%% 2 
\newcommand{\EXSY}{\myurls{https://onlinelibrary.wiley.com/journal/14680394}{Expert Systems}}                                                       		%% 3
\newcommand{\IEI}{\myurls{https://revistas.unal.edu.co/index.php/ingeinv}{Ingeniería e Investigación}}                                              		%% 4
\newcommand{\ITOR}{\myurls{https://onlinelibrary.wiley.com/journal/14753995}{International Transactions in Operational Research}}                   		%% 5
\newcommand{\ORIJ}{\myurls{https://www.springer.com/journal/12351}{Operational Research}}                                                           		%% 6
\newcommand{\IJPEDS}{\myurls{https://www.tandfonline.com/toc/gpaa20/current}{International Journal of Parallel, Emergent and Distributed Systems}}  		%% 7
\newcommand{\FLEX}{\myurls{https://www.springer.com/journal/10696}{Flexible Services and Manufacturing Journal}}                                    		%% 8
\newcommand{\EJIE}{\myurls{https://www.inderscience.com/jhome.php?jcode=ejie}{European Journal of Industrial Engineering}}                          		%% 9
\newcommand{\CIN}{\myurls{https://www.hindawi.com/journals/cin}{Computational Intelligence and Neuroscience}}                                       		%% 10
\newcommand{\SC}{\myurls{https://journals.sagepub.com/home/sci}{Science Progress}}                                                                  		%% 11
\newcommand{\CMPLX}{\myurls{https://www.hindawi.com/journals/complexity}{Complexity}}                                                               		%% 12
\newcommand{\FCS}{\myurls{https://www.springer.com/journal/11704}{Frontiers of Computer Science}}                                                   		%% 13
\newcommand{\SPJ}{\myurls{https://www.hindawi.com/journals/sp}{Scientific Programming}}                                                             		%% 14
\newcommand{\JEA}{\myurls{https://https://dl.acm.org/journal/jea}{ACM Journal of Experimental Algorithmics}}                                        		%% 15
\newcommand{\CIM}{\myurls{https://cis.ieee.org/publications/ci-magazine}{IEEE Computational Intelligence Magazine}}                                 		%% 16
\newcommand{\MINE}{\myurls{https://www.mdpi.com/journal/minerals}{Minerals}}                                                                        		%% 17
\newcommand{\ACISC}{\myurls{https://www.hindawi.com/journals/acisc}{Applied Computational Intelligence and Soft Computing}}                         		%% 18
\newcommand{\JIFS}{\myurls{https://www.iospress.com/catalog/journals/journal-of-intelligent-fuzzy-systems}{Journal of Intelligent \& Fuzzy Systems}}		%% 19
\newcommand{\IJSS}{\myurls{https://www.tandfonline.com/toc/tsyb20/current}{International Journal of Systems Science: Operations \& Logistics}}      		%% 20
\newcommand{\IJLSM}{\myurls{https://www.inderscience.com/jhome.php?jcode=ijlsm}{International Journal of Logistics Systems and Management}}         		%% 21
\newcommand{\PROC}{\myurls{https://www.mdpi.com/journal/processes}{Processes}}                                                                      		%% 22
\newcommand{\AOR}{\myurls{https://www.hindawi.com/journals/aor}{Advances in Operations Research}}                                                   		%% 23
\newcommand{\TETCI}{\myurls{https://cis.ieee.org/publications/t-emerging-topics-in-ci}{IEEE Transactions on Emerging Topics in Computational Intelligence}} %% 24
\newcommand{\PLOS}{\myurls{https://journals.plos.org/plosone}{PLoS One}}                                                                                    %% 25
\newcommand{\JS}{\myurls{https://www.springer.com/journal/11227}{Journal of Supercomputing}}                                                                %% 26
\newcommand{\MATHJ}{\myurls{https://www.mdpi.com/journal/mathematics}{Mathematics}}                                                                         %% 27
\newcommand{\KBS}{\myurls{https://www.sciencedirect.com/journal/knowledge-based-systems}{Knowledge-Based Systems}}                                          %% 28
\newcommand{\EAAI}{\myurls{https://www.sciencedirect.com/journal/engineering-applications-of-artificial-intelligence}{Engineering Applications of Artificial Intelligence}}                                                                                                                                              %% 29
\newcommand{\CAOR}{\myurls{https://www.sciencedirect.com/journal/computers-and-operations-research}{Computers \& Operations Research}}                      %% 30
\newcommand{\ENER}{\myurls{https://www.mdpi.com/journal/energies}{Energies}}                                                                        		%% 31
\newcommand{\ASC}{\myurls{https://www.sciencedirect.com/journal/applied-soft-computing}{Applied Soft Computing}}                                     		%% 32

\newcommand{\ejor}{8}
\newcommand{\ins}{9}
\newcommand{\exsy}{3}
\newcommand{\iei}{1}
\newcommand{\itor}{2}
\newcommand{\orij}{1}
\newcommand{\ijpeds}{1}
\newcommand{\flex}{9}
\newcommand{\ejie}{1}
\newcommand{\cin}{2}
\newcommand{\scj}{1}
\newcommand{\cmplx}{1}
\newcommand{\fcs}{2}
\newcommand{\spj}{2}
\newcommand{\jea}{2}
\newcommand{\cim}{2}
\newcommand{\mine}{2}
\newcommand{\acisc}{2}
\newcommand{\jifs}{1}
\newcommand{\ijss}{8}
\newcommand{\ijlsm}{1}
\newcommand{\proc}{3}
\newcommand{\aor}{2}
\newcommand{\tetci}{4}
\newcommand{\plos}{3}
\newcommand{\js}{1}
\newcommand{\mathj}{3}
\newcommand{\kbs}{1}
\newcommand{\eaai}{3}
\newcommand{\caor}{1}
\newcommand{\ener}{1}
\newcommand{\asc}{2}

%% Conferences
\newcommand{\ICPR}{\myurls{https://doi.org/10.1007/978-3-030-76307-7}{International Conference of Production Research}}
\newcommand{\CHCON}{\myurls{https://site.ieee.org/chilesur/ieee-chilecon-2023/}{IEEE Chilean Conference on Electrical, Electronics Engineering, Information and Communication Technologies}}


\newcommand{\icpr}{1}
\newcommand{\chcon}{1}

%% Sum of reviews
%% Papers
\newcommand{\SumPaper}{\the\numexpr \ejor + \ins + \exsy + \iei + \itor + \orij + \ijpeds + \flex + \ejie + \cin + \scj + \cmplx + \fcs + \spj + \jea + \cim + \mine + \acisc + \ijss + \jifs + \ijlsm + \proc + \aor + \tetci + \plos + \js + \mathj + \kbs + \caor + \eaai + \ener + \asc \relax}
%% Conferences
\newcommand{\SumConference}{\the\numexpr \chcon + \icpr \relax}




%% format
%%\newcommand{\journals}[1]{\FLEX{} (\flex), \EJOR{} (\ejor), \EXSY{} (\exsy), \INS{} (\ins), \CIN{} (\cin), \ITOR{} (\itor), \ORIJ{} (\orij), \IJPEDS{} (\ejie), \IEI{} (\iei), \EJIE{} (\ejie), \CMPLX~(\cmplx), \FCS~(\fcs), \SP~(\spj), \JEA~(\jea), \CIM~(\cim) {#1} \SC{} (\scj).}
\newcommand{\journals}[1]{\FLEX{} (\flex), \INS{} (\ins), \EJOR{} (\ejor), \IJSS{} (\ijss), \TETCI{} (\tetci), \EXSY{} (\exsy), \PROC{} (\proc), \MATHJ{} (\mathj), \PLOS{} (\plos), \EAAI{} (\eaai), \CIN{} (\cin), \MINE{} (\mine), \FCS{} (\fcs), \ACISC{} (\acisc), \SPJ{} (\spj), \AOR{} (\aor), \JEA{} (\jea), \ITOR{} (\itor), \CIM{} (\cim), \ASC{} (\asc), \IJPEDS{} (\ejie), \EJIE{} (\ejie), \ORIJ{} (\orij), \CMPLX{} (\cmplx), \IEI{} (\iei), \IJPEDS{} (\ijpeds), \IJLSM{} (\ijlsm), \JS{} (\js), \CAOR{} (\caor), \KBS{} (\kbs), \ENER{} (\ener), {#1} \SC{} (\scj).}
%\newcommand{\journals}[1]{}
\newcommand{\conferences}[1]{\CHCON{} (\chcon) {#1} \ICPR{} (\icpr).}

\es {
%% \newpage
   \section{Actividades de revisión}
   \cvitem{Artículos de revistas\\\SumPaper ~revisados}{\journals{y}}
   %\cvitem{}{\journalss{y}}
   \cvitem{Artículos de conferencias\\\SumConference ~revisados}{\conferences{y}}
}{
   \section{Peer review activities}
   \cvitem{Journal articles\\\SumPaper ~reviews}{\journals{and}}
   %\cvitem{}{\journalss{and}}
   \cvitem{Conference articles\\\SumConference ~reviews}{\conferences{and}}
}


%% Load metrics file
%% Metrics
{
\newread\file
\openin\file=citations.txt
\endlinechar-1%<- this removes the space at the end of each read line
\read\file to\totalGoogle
\read\file to\fiveYearsGoogle
\read\file to\citationGoogle
\read\file to\HindexGoogle
\read\file to\totalWOS
\read\file to\fiveYearsWOS
\read\file to\citationWOS
\read\file to\HindexWOS
\read\file to\totalScopus
\read\file to\fiveYearsScopus
\read\file to\citationScopus
\read\file to\HindexScopus
\read\file to\totalRG
\read\file to\fiveYearsRG
\read\file to\citationRG
\read\file to\HindexRG
\closein\file
%% \endgroup
\es{
  \section{Métricas de publicación}
     \cvitem{\aiGoogleScholar}{\myurls{https://scholar.google.es/citations?user=hIl4ixAAAAAJ\&hl}{Google Scholar}
             ~artículos (\totalGoogle{}), citas (\citationGoogle{}), índice h (\HindexGoogle{}), artículos en 5 años (\fiveYearsGoogle{}).}
     \cvitem{\aiClarivate}{\myurls{https://www.webofscience.com/wos/author/record/706245}{Clarivate}--\myurls{https://app.webofknowledge.com/author/\#/record/5462071}{WOS}
             ~artículos (\totalWOS{}), citas (\citationWOS{}), índice h (\HindexWOS{}), artículos en 5 años (\fiveYearsWOS{}).}
     \cvitem{\aiScopusSquare}{\myurls{https://www.scopus.com/authid/detail.uri?authorId=55097104500}{Scopus}
             ~~~~~~~~~~~~artículos (\totalScopus{}), citas (\citationScopus{}), índice h (\HindexScopus{}), artículos en 5 años (\fiveYearsScopus{}).}
     \cvitem{\aiResearchGate}{\myurls{https://www.researchgate.net/profile/Carlos\_Contreras\_Bolton}{ResearchGate}
             ~~~artículos (\totalRG{}), citas (\citationRG{}), índice h (\HindexRG{}), artículos en 5 años (\fiveYearsRG{}).}
}{
  \section{Publication metrics}
     \cvitem{\aiGoogleScholar}{\myurls{https://scholar.google.es/citations?user=hIl4ixAAAAAJ\&hl}{Google Scholar}
             ~articles (\totalGoogle{}), citations (\citationGoogle{}), h-index (\HindexGoogle{}), articles in 5-years (\fiveYearsGoogle{}).}
     \cvitem{\aiClarivate}{\myurls{https://www.webofscience.com/wos/author/record/706245}{Clarivate}--\myurls{https://app.webofknowledge.com/author/\#/record/5462071}{WOS}
             ~articles (\totalWOS{}), citations (\citationWOS{}), h-index (\HindexWOS{}), articles in 5-years (\fiveYearsWOS{}).}
     \cvitem{\aiScopusSquare}{\myurls{https://www.scopus.com/authid/detail.uri?authorId=55097104500}{Scopus}
             ~~~~~~~~~~~articles (\totalScopus{}), citations (\citationScopus{}), h-index (\HindexScopus{}), articles in 5-years (\fiveYearsScopus{}).}
     \cvitem{\aiResearchGate}{\myurls{https://www.researchgate.net/profile/Carlos\_Contreras\_Bolton}{ResearchGate}
             ~~articles (\totalRG{}), citations (\citationRG{}), h-index (\HindexRG{}), articles in 5-years (\fiveYearsRG{}).}
}
}


%% Load papers file
% Papers
\short
{
}{
\nociteSometido{*}
\es{\bibliographystyleSometido{spanish}}{\bibliographystyleSometido{english}}
\bibliographySometido{publicacionesSometido}                   % 'publications' es el

\nociteRevista{*}
\es{\bibliographystyleRevista{spanish}}{\bibliographystyleRevista{english}}
\bibliographyRevista{publicacionesRevista}                   % 'publications' es el
% nombre del archivo BibTeX

\nociteLibro{*}
\es{\bibliographystyleLibro{spanish}}{\bibliographystyleLibro{english}}
\bibliographyLibro{publicacionesLibro}

\nociteProceeding{*}
\es{\bibliographystyleProceeding{spanish}}{\bibliographystyleProceeding{english}}
\bibliographyProceeding{publicacionesProceeding}

\nociteConferencia{*}
\es{\bibliographystyleConferencia{spanish}}{\bibliographystyleConferencia{english}}
\bibliographyConferencia{publicacionesConferencia}      % 'publications'
%% es el nombre del archivo BibTeX
}



\end{document}
